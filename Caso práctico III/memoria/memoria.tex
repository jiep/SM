\documentclass[12pt,a4paper,twoside,openright,titlepage,final]{article}
\usepackage{fontspec}
\usepackage{amsmath}
\usepackage{amsfonts}
\usepackage{amssymb}
\usepackage{makeidx}
\usepackage{graphicx}
\usepackage[hidelinks,unicode=true]{hyperref}
\usepackage[spanish,es-nodecimaldot,es-lcroman,es-tabla,es-noshorthands]{babel}
\usepackage[left=3cm,right=2cm, bottom=4cm]{geometry}
\usepackage{natbib}
\usepackage{microtype}
\usepackage{ifdraft}
\usepackage{verbatim}
\usepackage[obeyDraft]{todonotes}
\ifdraft{
	\usepackage{draftwatermark}
	\SetWatermarkText{BORRADOR}
	\SetWatermarkScale{0.7}
	\SetWatermarkColor{red}
}{}
\usepackage{booktabs}
\usepackage{longtable}
\usepackage{calc}
\usepackage{array}
\usepackage{caption}
\usepackage{subfigure}
\usepackage{footnote}
\usepackage{url}

\usepackage{pdflscape}
\usepackage{minted}
\usepackage{blkarray}
\usepackage{pgf}
\usepackage{tikz}
\usetikzlibrary{arrows,automata}

%\setsansfont[Ligatures=TeX]{texgyreadventor}
%\setmainfont[Ligatures=TeX]{texgyrepagella}

\input{portada}

\author{José Ignacio Escribano}

\title{Caso práctico III: Análisis de resultados}

\setlength{\parindent}{0pt}

\begin{document}

\pagenumbering{alph}
\setcounter{page}{1}

\portada{Caso práctico III}{Simulación y Metaheurísticas}{Análisis de resultados}{José Ignacio Escribano}{Móstoles}

\listoftables
\thispagestyle{empty}
\newpage

\listoffigures
\thispagestyle{empty}
\newpage

\tableofcontents
\thispagestyle{empty}
\newpage


\pagenumbering{arabic}
\setcounter{page}{1}

\section{Introducción}

Este caso práctico veremos cómo simular una cadena de Markov en tiempo discreto.

\section{Resolución del caso práctico}

A continuación resolveremos cada una de las cuestiones planteadas.

\subsection{Cuestión 1}

Sabemos que la matriz de transición de la cadena de Markov tiene los siguientes valores:

\[ P = \left( \begin{array}{ccc}
0.1 & 0.4 & p_{13} \\
0.1 & 0.7 & p_{23} \\
0.3 & 0.1 & p_{33} \\
\end{array} \right) \]

donde $p_{13}, p_{23}$ y $p_{33}$ son valores desconocidos.\\

Puesto que la matriz $P$ debe ser una matriz estocástica, esto es, la suma de sus filas debe ser $1$, aplicamos esta restricción a cada una de las filas, quedando un sistema como el que sigue:

\[ \begin{array}{c}
0.1 + 0.4 + p_{13} = 1 \iff p_{13} = 0.5 \\
0.1 + 0.7 + p_{23} = 1 \iff p_{23} = 0.2 \\
0.3 + 0.1 + p_{33} = 1 \iff p_{33} = 0.6
\end{array} \]

Así pues, la matriz de transición de la cadena de Markov con espacio de estados $S = \{ \text{estodo0, estado1, estado2} \}$ queda de la siguiente forma:

\[ P = 
\begin{blockarray}{cccc}
&\text{estado0} & \text{estado1} & \text{estado2} \\
\begin{block}{c(ccc)}
\text{estado0} & 0.1 & 0.4 & 0.5 \\
\text{estado1} & 0.1 & 0.7 & 0.2 \\
\text{estado2} & 0.3 & 0.1 & 0.6 \\
\end{block}
\end{blockarray}
\]

La Figura~\ref{fig:grafo} muestra la matriz de transición en forma de grafo.\\

\begin{figure}[tbph!]
\label{fig:grafo}
\begin{center}
	\begin{tikzpicture}[->,>=stealth',shorten >=1pt,auto,node distance=6cm,
	semithick]
	
	\node[state] (A)                    {estado0};
	\node[state]         (B) [above right of=A] {estado1};
	\node[state]         (C) [below right of=B] {estado2};
	
	\path (A) edge[bend left ]  node {0.4} (B);
	\path (A) edge[ ]   node {0.5} (C);
	\path (B) edge[bend left]   node {0.2} (C);
	\path (B) edge[bend left]   node {0.1} (A);
	\path (C) edge[bend left]   node {0.1} (B);
	\path (C) edge[bend left]   node {0.3} (A);
	\path (A) edge[loop below]  node {0.1} (A);
	\path (C) edge[loop below]  node {0.6} (C);
	\path (B) edge[loop above]  node {0.7} (B);
	
	\end{tikzpicture}
\end{center}
\caption{Cadena de Markov en forma de grafo}
\end{figure}



\subsection{Cuestión 2}

Usamos el algoritmo visto en clase para generar transiciones de la cadena de Markov. El código implementado se puede ver en el Anexo~\ref{app:codigo}.\\

Haciendo $100$ réplicas con $100$ muestras de la cadena de Markov cada una, se obtienen los resultados de la Tabla~\ref{tbl:comp}.\\

\begin{table}[htbp!]
	\centering
	\caption{Media y desviacón típica de cada estado}
	\label{tbl:comp}
	\begin{tabular}{@{}ccc@{}}
		\toprule
		\textbf{Estado} & \textbf{Media} & \textbf{Desviación típica} \\ \midrule
		estado0         & 0.2933         & 0.035                      \\
		estado1         & 0.3068         & 0.042                      \\
		estado2         & 0.3998         & 0.041                      \\ \bottomrule
	\end{tabular}
\end{table}

Es decir, el computador estará en el 29.33\% del tiempo en el estado0, el 30.68\% en el estado1 y el 39.98\% en el estado2.\\



\section{Conclusiones}

En este caso práctico hemos visto cómo simular cadenas de Markov en tiempo discreto, una herramienta muy útil utilizada en diversas áreas.

\newpage

\appendix
\section{Código R utilizado}\label{app:codigo}

A continuación se muestra el código utilizado para la realización de este caso práctico.

\inputminted{r}{../codigo/caso_iii.R}


\end{document}