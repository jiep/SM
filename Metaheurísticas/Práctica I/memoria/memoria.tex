\documentclass[11pt,a4paper,twoside,openright,titlepage,final]{article}
\usepackage{fontspec}
\usepackage{amsmath}
\usepackage{amsfonts}
\usepackage{amssymb}
\usepackage{makeidx}
\usepackage{graphicx}
\usepackage[hidelinks,unicode=true]{hyperref}
\usepackage[spanish,es-nodecimaldot,es-lcroman,es-tabla,es-noshorthands]{babel}
\usepackage[left=3cm,right=2cm, bottom=4cm]{geometry}
\usepackage{natbib}
\usepackage{microtype}
\usepackage{ifdraft}
\usepackage{verbatim}
\usepackage[obeyDraft]{todonotes}
\ifdraft{
	\usepackage{draftwatermark}
	\SetWatermarkText{BORRADOR}
	\SetWatermarkScale{0.7}
	\SetWatermarkColor{red}
}{}
\usepackage{booktabs}
\usepackage{longtable}
\usepackage{calc}
\usepackage{array}
\usepackage{caption}
\usepackage{subfigure}
\usepackage{footnote}
\usepackage{url}
\usepackage{tikz}
\usepackage{pdflscape}
\usepackage{minted}

\setsansfont[Ligatures=TeX]{texgyreadventor}
\setmainfont[Ligatures=TeX]{texgyrepagella}

\input{portada}

\author{José Ignacio Escribano}

\setlength{\parindent}{0pt}

\begin{document}

\pagenumbering{alph}
\setcounter{page}{1}

\portada{Práctica I}{Simulación y Metaheurísticas}{Problemas e instancias}{José Ignacio Escribano}{Móstoles}

\listoftables
\thispagestyle{empty}
\newpage

\tableofcontents
\thispagestyle{empty}
\newpage


\pagenumbering{arabic}
\setcounter{page}{1}

\section{Introducción}

El problema del capacited p-hub consiste en determinar los $p$ centros (hubs) que actúan como servidores y conectar todos los clientes a uno de los servidores de forma que se minimice la suma de las distancias de los clientes a los servidores. Además, cada servidor sólo puede puede servir una determinada cantidad de recursos, y cada cliente tiene una demanda que deben ser satisfechas. De forma matemática, el modelo se formula así:

\begin{alignat*}{2}
\text{Minimizar }   & \sum_{i} \sum_{j} d_{ij}x_{ij}  \\
\text{s.a. } &  \sum_{j} x_{ij} = 1 \quad \forall i\\
&  x_{ij} \leq y_j \quad \forall i,j\\
& \sum_{j} y_j = p \\
& x_{ij}, y_j \in \{0,1\} \\
& \sum_{i} x_{ij} b_j \leq c
\end{alignat*}

donde

\begin{itemize}
	\item $x_{ij}$ es $1$ si hay arista entre el cliente $i$ y el servidor $j$.
	\item $y_i$ es $1$  si $j$ es servidor, $0$ en caso contrario.
	\item $d_{ij}$ es la distancia entre el nodo $i$ y el $j$.
	\item $b_j$ es la demanda del nodo $j$
	\item $c$ es la capacidad del servidor
\end{itemize}

\section{Resolución de la práctica}

A continuación, resolveremos las cuestiones planteadas en la práctica.

\subsection{Estructuras de datos de las instancias}

Las instancias para el problema del \textit{p-hub} se encuentran en la clase \texttt{InstanciaPHub}, cuyos atributos son los siguientes:

\begin{enumerate}
	\item nodos: número de nodos de la instancia.
	\item servidores: número de servidores de la instancia (p).
	\item distancia: matriz cuadrada que contiene la distancia entre cada par de nodos, calculada a partir de las coordenadas de cada centro usando la distancia euclídea, esto es,
	
	\begin{equation}
	\label{eq:distancia}
	d(\mathbf{x} =(x_1,\dots, x_n), \mathbf{y} = (y_1,\dots, y_n)) = \sqrt{\sum_{i=1}^{n} (x_i - y_i)^2 }
	\end{equation}
	
	\item demanda: vector con la demanda de cada nodo.
	\item capacidad: capacidad de los servidores.
\end{enumerate}

\subsection{Carga de las instancias}

Las instancias se cargan en memoria gracias al procedimiento \texttt{leerInstancia} de la clase \texttt{PHub}. Este procedimiento recibe como parámetro el nombre del fichero y va leyendo línea a línea los datos del fichero. Estos datos se guardan en sus respectivas estructuras de datos, que se pasan al constructor de la clase \texttt{InstanciaPHub}, que ``genera'' el objeto. Este procedimiento también se encarga de almacenar las posiciones de los centros y las almacena en memoria, para llamar a la función \texttt{calcularDistancias} de la clase \texttt{PHub}, que se encarga de devolver una matriz de distancias haciendo uso de la Ecuación~\ref{eq:distancia}.

\subsection{Estructura de la solución}

Cada solución se guarda en un objeto de la clase \texttt{Solución}, que contiene los siguientes atributos:

\begin{enumerate}
	\item solución: vector de booleanos que indica \texttt{true}, si el nodo $i$ es un servidor, \texttt{false} en caso contrario.
	\item matrizAdyacencia: matriz de booleanos, que contiene la matriz de adyacencia del grafo, esto es, \texttt{true}, si hay una arista entre el nodo $i$ y $j$, \texttt{false} en caso contrario.
	\item objetivo: valor de la función objetivo, es decir, la suma de las distancias de las aristas existentes entre los clientes y los objetivos. 
\end{enumerate}

\subsection{Cálculo de la función objetivo}

El cálculo de la función objetivo se realiza en la función \texttt{calcularObjetivo} de la clase \texttt{Utils}, que dada una solución, su matriz de adyacencia y de distancias, calcula su valor.

El código es el siguiente:

\begin{minted}[linenos=true, numberblanklines=true, breaklines=true]{java}
public static double calcularObjetivo(boolean[] solucion, boolean[][] matrizAdyacencia, double[][] distancia) {

    double obj = 0;
    for (int i = 0; i < matrizAdyacencia.length; i++) {
        for (int j = 0; j <= i; j++) {
            // Si hay conexión entre los nodos, sumamos
            if (matrizAdyacencia[i][j]) {
                obj += distancia[i][j];
            }
        }
    }
    return obj;
}
\end{minted}

Esta función recorre la matriz de adyacencia (notar que es simétrica), y si existe una arista entre el nodo $i$ y el $j$, suma la distancia entre estos dos nodos a la variable \texttt{obj}, que se devuelve.

\subsection{Solución aleatoria de una instancia}

Para generar una solución aleatoria de una instancia está la función \texttt{generarSoluciónAleatoria} de la clase \texttt{InstanciaPHub}. El código es el siguiente:

\begin{minted}[linenos=true, numberblanklines=true, breaklines=true]{java}
public Solución generarSoluciónAleatoria() {

    Solución s1 = null;

    Random x = new Random();

    // Para poder reproducir los resultados
    x.setSeed(0);

    do{
        x = new Random();

        int nodos = this.getNodos();
        int num_servidores = 0;

        boolean[] sol = new boolean[nodos];
        boolean[][] ady = new boolean[nodos][nodos];

        // Elegimos al azar los servidores
        while (num_servidores < this.getServidores()) {
            int aleatorio = 0;
        
            do {
                aleatorio = x.nextInt(nodos);
            } while (sol[aleatorio]);

            sol[aleatorio] = true;

            num_servidores++;
        }

        // Generamos las aristas
        for (int i = 0; i < nodos; i++) {
            if (!sol[i]) {
                // Seleccionamos el servidor más cercano que nos encontremos
                int serv = Utils.seleccionarServidor(i, sol, this.getDistancia());

                ady[i][serv] = true;
                ady[serv][i] = true;
            }

         }

         s1 = new Solución(sol, ady, this.getDistancia());

    }while(!Utils.esSoluciónVálida(s1, this));

    return s1 ;
}
\end{minted}

En las líneas 20 a 30 se seleccionan al azar los $p$ nodos que actuarán como servidores. Para ello, se generan números aleatorios en el intervalo discreto $[0, \text{número de nodos})$. Una vez que se tienen los nodos que actuarán como servidores, se selecciona a qué servidor se conectará cada cliente. Para ello, se busca el servidor más cercano (el que tiene una menor distancia euclídea) al cliente. De esto se encarga la función \texttt{seleccionarServidor} de la clase \texttt{Utils}. Una vez se tiene el servidor más cercano al cliente, se establece a \texttt{true} la posición de la matriz de adyacencia entre el servidor asignado y el cliente (notar que la matriz de adyacencia es simétrica). Con todos estos datos, se crea un objeto de la clase \texttt{Solución} y se comprueba si es una solución válida, es decir, si cumple las restricciones del problema. Si lo es, se devuelve la solución, y en caso contrario se repite el proceso anterior hasta que se obtenga una solución válida.\\

El código de la función \texttt{esSoluciónVálida} de la clase \texttt{Utils} es el siguiente:

\begin{minted}[linenos=true, numberblanklines=true, breaklines=true]{java}
public static boolean esSoluciónVálida(Solución s, InstanciaPHub instancia) {
    boolean esValida = true;
    boolean[][] ady = s.getMatrizAdyacencia();
    boolean[] sol = s.getSolucion();
    int capacidad = instancia.getCapacidad();
    int demanda[] = new int[instancia.getServidores()];
    int cont = 0;
	
    // Comprobamos si se cumple el criterio de la demanda
    for (int i = 0; i < sol.length; i++) {
        if (sol[i]) {
            for (int j = 0; j < sol.length; j++) {
                if (ady[i][j]) {
                    demanda[cont] += instancia.getDemanda()[j];
                }
            }
            cont++;
        } 
    }
	
    for (int i = 0; i < demanda.length; i++) {
        if (demanda[i] > capacidad) {
		    esValida = false;
        }
    }
	
    return esValida;
	
}
\end{minted}

En las líneas 9 a 19 se recorre el vector solución (que contiene \texttt{true}, si el nodo $i$ es un servidor, y \texttt{false} en caso contrario) para calcular la demanda de ese servidor. Este proceso se repite para todos los servidores. Una vez calculado, se comprueba si existe algún servidor que no cumple la demanda, esto es, si la suma de las demandas de cada cliente es menor o igual a la capacidad del servidor. Si se comprueba que todos los servidores tienen una demanda menor a la capacidad se devuelve \texttt{true}. En caso contrario \texttt{false}.

\subsection{Programa principal}

El programa principal se encarga de generar $1000$ soluciones aleatorias de cada una de las instancias, y devolviendo el valor de la función objetivo y la estructura de la mejor solución de cada instancia.\\

Los valores de la mejor solución de cada instancia se muestran en la Tabla~\ref{tbl:mejores}.\\

\begin{table}[htbp!]
	\centering
	\caption{Mejores soluciones de cada instancia}
	\label{tbl:mejores}
	\begin{tabular}{@{}ccc@{}}
		\toprule
		Instancia & Función objetivo & Servidores               \\ \midrule
		1         & 751.8302         & {[}1, 2, 17, 19, 34{]}   \\
		2         & 855.1385         & {[}9, 16, 20, 34, 39{]}  \\
		3         & 762.2916         & {[}16, 22, 26, 28, 33{]} \\
		4         & 776.3040         & {[}26, 28, 34, 35, 41{]} \\
		5         & 686.0421         & {[}3, 29, 35, 43, 50{]}  \\
		6         & 689.6972         & {[}4, 22, 29, 40, 49{]}  \\
		7         & 815.7641         & {[}17, 19, 38, 42, 49{]} \\
		8         & 824.0950         & {[}4, 19, 34, 46, 49{]}  \\
		9         & 860.1325         & {[}3, 13, 24, 37, 49{]}  \\
		10        & 736.3474         & {[}7, 11, 22, 30, 38{]}  \\ \bottomrule
	\end{tabular}
\end{table}

Notar que se han omitido la matriz de adyacencia por claridad. Las soluciones completas se pueden encontrar en el fichero \texttt{mejores\_soluciones.txt} adjunto a esta memoria.


\section{Conclusiones}

En esta práctica nos hemos introducido en el problema p-hub. Hemos cargado instancias de este problema en memoria y hemos generado soluciones aleatorias de cada una de las instancias. En último lugar hemos simulado 1000 soluciones de cada instancia y hemos seleccionado la mejor (la que tenía menor valor de la función objetivo).

\end{document}