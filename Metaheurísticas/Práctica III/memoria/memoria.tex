\documentclass[12pt,a4paper,twoside,openright,titlepage,final]{article}
\usepackage{fontspec}
\usepackage{amsmath}
\usepackage{amsfonts}
\usepackage{amssymb}
\usepackage{makeidx}
\usepackage{graphicx}
\usepackage[hidelinks,unicode=true]{hyperref}
\usepackage[spanish,es-nodecimaldot,es-lcroman,es-tabla,es-noshorthands]{babel}
\usepackage[left=3cm,right=2cm, bottom=4cm]{geometry}
\usepackage{natbib}
\usepackage{microtype}
\usepackage{ifdraft}
\usepackage{verbatim}
\usepackage[table,xcdraw]{xcolor}
\usepackage[obeyDraft]{todonotes}
\ifdraft{
	\usepackage{draftwatermark}
	\SetWatermarkText{BORRADOR}
	\SetWatermarkScale{0.7}
	\SetWatermarkColor{red}
}{}
\usepackage{booktabs}
\usepackage{longtable}
\usepackage{calc}
\usepackage{array}
\usepackage{caption}
\usepackage{subfigure}
\usepackage{footnote}
\usepackage{url}
\usepackage{tikz}
\usepackage{pdflscape}
\usepackage{minted}
\usepackage{multirow}
\usepackage{pdflscape}


%\setsansfont[Ligatures=TeX]{texgyreadventor}
%\setmainfont[Ligatures=TeX]{texgyrepagella}

\input{portada}

\author{José Ignacio Escribano}

\setlength{\parindent}{0pt}

\begin{document}

\pagenumbering{alph}
\setcounter{page}{1}

\portada{Práctica III}{Simulación y Metaheurísticas}{Metaheurísticas}{José Ignacio Escribano}{Móstoles}

\listoftables
\thispagestyle{empty}
\newpage

\tableofcontents
\thispagestyle{empty}
\newpage

\pagenumbering{arabic}
\setcounter{page}{1}

\section{Introducción}

En esta práctica implementaremos una de las metaheurísticas planteadas para el problema del p-hubs.


\section{Resolución de la práctica}

A continuación mostraremos la implementación del recocido simulado para el problema del p-hub. El código se encuentra en el método estático \texttt{recocidoSimulado} de la clase \texttt{Práctica3}.

\begin{minted}[linenos=true, numberblanklines=true, breaklines=true]{java}
static Solución recocidoSimulado(Enfriamiento.TIPOS_ENFRIAMIENTO tipo, Solución sol_ini, InstanciaPHub instancia,
int temp_ini, int nrep, double alpha, long tiempo) {

    double tk = 0;

    Solución x = sol_ini;

    int i = 1;

    switch (tipo) {
        case BOLTZMANN: tk = Enfriamiento.crierioBoltzmann(temp_ini, alpha, i); break;
        case CAUCHY: tk = Enfriamiento.esquemaCauchy(temp_ini, alpha, i); break;
        case LUNDYyMEES: tk = Enfriamiento.esquemaCauchy(temp_ini, alpha, i); break;
        case DESCENSO_GEOMETRICO: tk = Enfriamiento.crierioBoltzmann(temp_ini, alpha, i);
    }

    long fin = System.currentTimeMillis() + tiempo;

    double delta;

    do {
        int m = 0;
        do {
            // Generamos número aleatorio entre 0 y número de soluciones de la vecindad - 1
            Random r = new Random();
            List<Solución> vecindad = PHub.generarVecindad(x, instancia);
            int n = r.nextInt(vecindad.size());

            Solución y = vecindad.get(n);

            delta = y.getObjetivo() - x.getObjetivo();

            if (delta < 0) {
                x = y;
            } else {
                double u = r.nextDouble();
                if (u <= Math.exp(-delta / tk)) {
                    x = y;
                }
            }
            m++;
        } while (m != nrep);
        
        switch (tipo) {
            case BOLTZMANN: tk = Enfriamiento.crierioBoltzmann(temp_ini, alpha, i); break;
            case CAUCHY: tk = Enfriamiento.esquemaCauchy(temp_ini, alpha, i); break;
            case LUNDYyMEES: tk = Enfriamiento.esquemaCauchy(temp_ini, alpha, i); break;
            case DESCENSO_GEOMETRICO: tk = Enfriamiento.crierioBoltzmann(temp_ini, alpha, i);
        }
        i++;
    } while (System.currentTimeMillis() <= fin);

    return x;
}
\end{minted}

Esta función recibe un tipo de enfriamiento, una solución inicial, una instancia, una temperatura inicial, un número de repeticiones, un valor (parámetro de la función de enfriamiento) y el tiempo de ejecución del programa, en milisegundos.\\

En la línea 6 se establece que \texttt{x} es la solución inicial, y desde la línea 10 a la 16 se establece qué función de enfriamiento se utilizará para el recocido simulado.\\

Se elige una solución al azar de la vecindad de \texttt{x}, que llamaremos \texttt{y}, y se calcula la diferencia entre la solución nueva y la inicial. A este valor lo llamemos \texttt{delta}. Si \texttt{delta} es menor que cero, es decir $\delta = f(y) - f(x) < 0 \iff f(y) < f(x)$, es decir, se mejora la solución actual y se guarda. En caso contrario (si no se mejora la solución) se genera un número aleatorio en el intervalo $(0,1)$. Si este número aleatorio es menor que $e^{-\delta/t_k}$, se guarda la solución \texttt{y}. Notar que $t_k$ es el valor de la temperatura en la iteración $k$. Este proceso se repite \texttt{nrep} veces, y todo lo anterior se repite durante el tiempo especificado. Éste es nuestro criterio de parada.\\

En cuanto la funciones de enfriamiento se ha definido una clase \texttt{Enfriamiento} dentro de la clase \texttt{Práctica3}. Se han implementado cuatro funciones de enfriamiento, que vienen dadas por las siguientes ecuaciones:

\begin{itemize}
\item Descenso geométrico: $t_{i+1} = \alpha t_i, \ \alpha \in [0.8, 0.99]$
\item Criterio de Boltzmann: $t_i = \dfrac{t_0}{1 + \log i}$
\item Esquema de Cauchy $t_i = \dfrac{t_0}{1+i}$
\item Lundy y Mees:  $t_{i+1} = \dfrac{t_i}{1 + \beta t_i}$, con $\beta$ muy pequeño
\end{itemize}

El código del criterio de Boltzmann se muestra a continuación:

\begin{minted}[linenos=true, numberblanklines=true, breaklines=true]{java}
public static double crierioBoltzmann(double t0, double alpha, int i) {
    return t0 / (1 + Math.log(i));
}
\end{minted}

Esta función recibe tres parámetros: la temperatura inicial, el valor del parámetro $\alpha$ y la iteración i. Se devuelve directamente el valor dado por la fórmula indicada anteriormente.\\ 

De forma similar se implementan los demás métodos.

\subsection{Análisis de los resultados}

Para evaluar este método ejecutamos durante 1 minuto (60000 milisegundos) cada una de las instancias con cada uno de las funciones de enfriamiento.\\

Los parámetros usados han sido los siguientes:

\begin{itemize}
	\item Temperatura inicial = 10000
	\item $\alpha = \begin{cases}
	0.99, \text{ si es descenso geométrico}\\
	0.001, \text{ si es Lundi y Mees}
	\end{cases}$
	\item Número de repeticiones = 250
	\item Tiempo de ejecución (condición de parada) = 60000 (1 minuto por instancia)
\end{itemize}

Los resultados de la ejecución se pueden ver en la Tabla~\ref{tbl:resultados}. En amarillo se muestra el mejor valor de la función objetivo de cada instancia.\\

\begin{table}[htbp!]
	\centering
	\caption{Resultados de la ejecución del recocido simulado}
	\label{tbl:resultados}
\begin{tabular}{@{}cccccc@{}}
	\toprule
	& \multicolumn{4}{c}{Función de enfriamiento}                                                                                                                                                                                                             &                         \\ \cmidrule(lr){2-5}
	\multirow{-2}{*}{Instancia} & \begin{tabular}[c]{@{}c@{}}Descenso\\ geométrico\end{tabular} & \begin{tabular}[c]{@{}c@{}}Criterio de\\ Boltzmann\end{tabular} & \begin{tabular}[c]{@{}c@{}}Esquema \\ de Cauchy\end{tabular} & \begin{tabular}[c]{@{}c@{}}Lundi y\\ Meer\end{tabular} & \multirow{-2}{*}{Media} \\ \midrule
	1                           & 932.4007                                                      & 1160.496                                                        & 991.9606                                                     & \cellcolor[HTML]{FFFE65}831.4168                       & 979.0685                \\
	2                           & 986.0779                                                      & 1054.8835                                                       & 1105.693                                                     & \cellcolor[HTML]{FFFE65}883.8535                       & 1007.6270               \\
	3                           & 974.4974                                                      & 994.4498                                                        & 993.2027                                                     & \cellcolor[HTML]{FFFE65}825.3837                       & 946.8834                \\
	4                           & 1039.086                                                      & \cellcolor[HTML]{FFFE65}980.6139                                & 1150.0283                                                    & 1024.4868                                              & 1048.5538               \\
	5                           & 929.1119                                                      & 1037.8508                                                       & 949.8733                                                     & \cellcolor[HTML]{FFFE65}712.1517                       & 907.2469                \\
	6                           & 977.0404                                                      & \cellcolor[HTML]{FFFE65}787.7134                                & 1056.724                                                     & 870.6946                                               & 923.0431                \\
	7                           & 1021.4352                                                     & 976.2935                                                        & 982.2888                                                     & \cellcolor[HTML]{FFFE65}945.6088                       & 981.4066                \\
	8                           & 1056.8216                                                     & 1057.9631                                                       & 825.4254                                                     & \cellcolor[HTML]{FFFE65}977.5117                       & 979.4304                \\
	9                           & 981.1257                                                      & 1074.5005                                                       & 1041.2094                                                    & \cellcolor[HTML]{FFFE65}896.6903                       & 998.3815                \\
	10                          & 836.5384                                                      & 832.3738                                                        & 950.8749                                                     & \cellcolor[HTML]{FFFE65}757.631                        & 844.3545                \\ \midrule
	Media                       & 973.4135                                                      & 995.7138                                                        & 1004.7280                                                    & 872.5429                                               &                         \\ \bottomrule
\end{tabular}
\end{table}


Se puede observar que no se obtiene ninguna mejor instancia con el método del descenso geométrico, ni con el esquema de Cauchy, en las instancias 4 y 6, el mejor método es el criterio de Boltzmann, y en las instancias 1, 2, 3, 5, 7, 8, 9 y 10 se obtiene la mejor solución con el método de Landi y Mees. Es decir, con el método de Lundi y Meer se obtienen 8  mejores instancias, con el criterio de Boltzmann se obtienen 2 instancias y con el esquema de Cauchy y el descenso geométrico no se obtiene ninguna instancia. Además, el método de Lundi y Meer tiene la mejor media de valores de la función objetivo. Parece que el mejor de los cuatro métodos es el Lundi y Meer (con los parámetros definidos anteriormente).\\

Necesitamos conocer cuál es mejor valor de $\beta$ que minimiza el valor de las instancias. Para ello, repetimos el proceso anterior para distintos parámetros de $\beta$, desde $1$ hasta $10^{-10}$. Los resultados se pueden ver en la Tabla~\ref{tbl:resultados_landi}.\\

Se puede observar que el parámetro con mayor número de instancias con la solución menor es $\beta=0.1$, que parece un buen valor para este método, aunque se deberían realizar prueba de $\beta$ en el intervalo $[0.1, 1]$ para comprobar si se consiguen soluciones mejores que las actuales.\\

\begin{landscape}
	\vspace*{\fill}
	\topskip0pt    
	\begin{table}[htbp!]
		\centering
		\caption{Resultados de la ejecución del recocido con el método simulado con la función de enfriamiento de Lundi y Meer para distintos valores del parámetro $\beta$}
		\label{tbl:resultados_landi}
		\resizebox{\linewidth}{!}{%
			\begin{tabular}{@{}cccccccccccc@{}}
				\toprule
				Instancia & $\beta = 1$                      & $\beta = 0.1$                    & $\beta = 0.01$                   & $\beta = 0.001$ & $\beta = 10^{-4}$ & $\beta = 10^{-5}$ & $\beta = 10^{-6}$ & $\beta = 10^{-7}$ & $\beta = 10^{-8}$ & $\beta = 10^{-9}$ & $\beta = 10^{-10}$ \\ \midrule
				1         & \cellcolor[HTML]{FFFE65}708.4036 & \cellcolor[HTML]{FFFE65}708.4036 & \cellcolor[HTML]{FFFE65}708.4036 & 819.0216        & 888.7935          & 1537.5536         & 983.174           & 941.2269          & 960.708           & 1130.6897         & 1031.907           \\
				2         & \cellcolor[HTML]{FFFE65}829.3524 & \cellcolor[HTML]{FFFE65}829.3524 & 845.3045                         & 959.4793        & 871.7137          & 997.4544          & 1044.877          & 1119.9853         & 1341.9047         & 1103.3533         & 889.3522           \\
				3         & \cellcolor[HTML]{FFFE65}758.2295 & \cellcolor[HTML]{FFFE65}758.2295 & 781.0541                         & 866.4344        & 827.2772          & 857.758           & 931.0008          & 947.5278          & 1010.2579         & 923.8711          & 909.0433           \\
				4         & 851.259                          & \cellcolor[HTML]{FFFE65}764.0963 & 771.3152                         & 852.2911        & 878.914           & 1113.301          & 929.5215          & 1178.9504         & 912.6937          & 1191.0961         & 1149.1484          \\
				5         & \cellcolor[HTML]{FFFE65}671.1279 & 681.3047                         & 671.3733                         & 862.9865        & 785.1307          & 1035.2035         & 802.9528          & 904.586           & 989.3504          & 890.3637          & 851.9101           \\
				6         & \cellcolor[HTML]{FFFE65}663.1406 & \cellcolor[HTML]{FFFE65}663.1406 & 682.945                          & 738.7844        & 857.4963          & 726.1366          & 826.3958          & 1178.5185         & 814.0449          & 883.9562          & 796.0767           \\
				7         & 807.82                           & \cellcolor[HTML]{FFFE65}793.2306 & 798.0115                         & 826.2132        & 876.9625          & 887.4507          & 981.5936          & 1119.5116         & 832.6994          & 941.605           & 958.4476           \\
				8         & 829.1049                         & \cellcolor[HTML]{FFFE65}822.5995 & 827.965                          & 846.9209        & 876.0916          & 962.5802          & 1004.3007         & 923.5164          & 954.9392          & 887.6405          & 917.5369           \\
				9         & 857.4355                         & \cellcolor[HTML]{FFFE65}840.0549 & 873.247                          & 919.9782        & 1255.2681         & 912.541           & 1211.4791         & 1461.8266         & 962.1836          & 1066.2949         & 995.2715           \\
				10        & 824.7164                         & \cellcolor[HTML]{FFFE65}722.024  & 742.7818                         & 958.8561        & 914.895           & 762.6091          & 909.6664          & 876.8673          & 1147.981          & 794.522           & 820.6012           \\ \bottomrule
			\end{tabular}
		}
	\end{table}
	\vspace*{\fill}
\end{landscape}

\section{Conclusiones}

En esta práctica hemos visto cómo implementar una de las metaheurísticas vistas en clase: el recocido simulado. También hemos visto que existen diversos métodos para establecer el enfriamiento, dando todos ellos a distintos resultados, ya que la mayoría es una familia uniparamétrica. Esto hace que sea indispensable probar con distintas metaheurísticas (y sus distintos parámetros) y escoger la que mejor resultados obtenga para el problema en concreto.


\end{document}