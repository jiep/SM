\documentclass[12pt,a4paper,twoside,openright,titlepage,final]{article}
\usepackage{fontspec}
\usepackage{amsmath}
\usepackage{amsfonts}
\usepackage{amssymb}
\usepackage{makeidx}
\usepackage{graphicx}
\usepackage[hidelinks,unicode=true]{hyperref}
\usepackage[spanish,es-nodecimaldot,es-lcroman,es-tabla,es-noshorthands]{babel}
\usepackage[left=3cm,right=2cm, bottom=4cm]{geometry}
\usepackage{natbib}
\usepackage{microtype}
\usepackage{ifdraft}
\usepackage{verbatim}
\usepackage[table,xcdraw]{xcolor}
\usepackage[obeyDraft]{todonotes}
\ifdraft{
	\usepackage{draftwatermark}
	\SetWatermarkText{BORRADOR}
	\SetWatermarkScale{0.7}
	\SetWatermarkColor{red}
}{}
\usepackage{booktabs}
\usepackage{longtable}
\usepackage{calc}
\usepackage{array}
\usepackage{caption}
\usepackage{subfigure}
\usepackage{footnote}
\usepackage{url}
\usepackage{tikz}
\usepackage{pdflscape}
\usepackage{minted}
\usepackage{multirow}
\usepackage{pdflscape}


%\setsansfont[Ligatures=TeX]{texgyreadventor}
%\setmainfont[Ligatures=TeX]{texgyrepagella}

\input{portada}

\author{José Ignacio Escribano}

\setlength{\parindent}{0pt}

\begin{document}

\pagenumbering{alph}
\setcounter{page}{1}

\portada{Práctica III}{Simulación y Metaheurísticas}{Metaheurísticas}{José Ignacio Escribano}{Móstoles}

\listoftables
\thispagestyle{empty}
\newpage

\listoffigures
\thispagestyle{empty}
\newpage

\tableofcontents
\thispagestyle{empty}
\newpage


\pagenumbering{arabic}
\setcounter{page}{1}

\section{Introducción}

En esta práctica implementaremos una de las metaheurísticas planteadas para el problema del p-hubs.


\section{Resolución de la práctica}

A continuación mostraremos la implementación del recocido simulado para el problema del p-hub. El código se encuentra en el método estático \texttt{recocidoSimulado} de la clase \texttt{Práctica3}.

\begin{minted}[linenos=true, numberblanklines=true, breaklines=true]{java}
static Solución recocidoSimulado(Enfriamiento.TIPOS_ENFRIAMIENTO tipo, Solución sol_ini, InstanciaPHub instancia,
int temp_ini, int nrep, double alpha, long tiempo) {

    double tk = 0;

    Solución x = sol_ini;

    int i = 1;

    switch (tipo) {
        case BOLTZMANN: tk = Enfriamiento.crierioBoltzmann(temp_ini, alpha, i); break;
        case CAUCHY: tk = Enfriamiento.esquemaCauchy(temp_ini, alpha, i); break;
        case LUNDYyMEES: tk = Enfriamiento.esquemaCauchy(temp_ini, alpha, i); break;
        case DESCENSO_GEOMETRICO: tk = Enfriamiento.crierioBoltzmann(temp_ini, alpha, i);
    }

    long fin = System.currentTimeMillis() + tiempo;

    double delta;

    do {
        int m = 0;
        do {
            // Generamos número aleatorio entre 0 y número de soluciones de la vecindad - 1
            Random r = new Random();
            List<Solución> vecindad = PHub.generarVecindad(x, instancia);
            int n = r.nextInt(vecindad.size());

            Solución y = vecindad.get(n);

            delta = y.getObjetivo() - x.getObjetivo();

            if (delta < 0) {
                x = y;
            } else {
                double u = r.nextDouble();
                if (u <= Math.exp(-delta / tk)) {
                    x = y;
                }
            }
            m++;
        } while (m != nrep);
        
        switch (tipo) {
            case BOLTZMANN: tk = Enfriamiento.crierioBoltzmann(temp_ini, alpha, i); break;
            case CAUCHY: tk = Enfriamiento.esquemaCauchy(temp_ini, alpha, i); break;
            case LUNDYyMEES: tk = Enfriamiento.esquemaCauchy(temp_ini, alpha, i); break;
            case DESCENSO_GEOMETRICO: tk = Enfriamiento.crierioBoltzmann(temp_ini, alpha, i);
        }
        i++;
    } while (System.currentTimeMillis() <= fin);

    return x;
}
\end{minted}

Esta función recibe un tipo de enfriamiento, una solución inicial, una instancia, una temperatura inicial, un número de repeticiones, un valor (parámetro de la función de enfriamiento) y el tiempo de ejecución del programa, en milisegundos.\\

En la línea 6 se establece que \texttt{x} es la solución inicial, y desde la línea 10 a la 16 se establece qué función de enfriamiento se utilizará para el recocido simulado.\\

Se elige una solución al azar de la vecindad de \texttt{x}, que llamaremos \texttt{y}, y se calcula la diferencia entre la solución nueva y la inicial. A este valor lo llamemos \texttt{delta}. Si \texttt{delta} es menor que cero, es decir $\delta = f(y) - f(x) < 0 \iff f(y) < f(x)$, es decir, se mejora la solución actual y se guarda. En caso contrario (si no se mejora la solución) se genera un número aleatorio en el intervalo $(0,1)$. Si este número aleatorio es menor que $e^{-\delta/t_k}$, se guarda la solución \texttt{y}. Notar que $t_k$ es el valor de la temperatura en la iteración $k$. Este proceso se repite \texttt{nrep} veces, y todo lo anterior se repite durante el tiempo especificado. Éste es nuestro criterio de parada.\\

En cuanto la funciones de enfriamiento se ha definido una clase \texttt{Enfriamiento} dentro de la clase \texttt{Práctica3}. Se han implementado cuatro funciones de enfriamiento, que vienen dadas por las siguientes ecuaciones:

\begin{itemize}
\item Descenso geométrico: $t_{i+1} = \alpha t_i, \ \alpha \in [0.8, 0.99]$
\item Criterio de Boltzmann: $t_i = \dfrac{t_0}{1 + \log i}$
\item Esquema de Cauchy $t_i = \dfrac{t_0}{1+i}$
\item Lundy y Mees:  $t_{i+1} = \dfrac{t_i}{1 + \beta t_i}$, con $\beta$ muy pequeño
\end{itemize}

El código del criterio de Boltzmann se muestra a continuación:

\begin{minted}[linenos=true, numberblanklines=true, breaklines=true]{java}
public static double crierioBoltzmann(double t0, double alpha, int i) {
    return t0 / (1 + Math.log(i));
}
\end{minted}

Esta función recibe tres parámetros: la temperatura inicial, el valor del parámetro $\alpha$ y la iteración i. Se devuelve directamente el valor dado por la fórmula indicada anteriormente.\\ 

De forma similar se implementan los demás métodos.

\subsection{Análisis de los resultados}

Para evaluar este método ejecutamos durante 1 minuto (60 milisegundos) cada una de las instancias con cada uno de las funciones de enfriamiento.\\

Los parámetros usados han sido los siguientes:

\begin{itemize}
	\item Temperatura inicial = 10000
	\item $\alpha = \begin{cases}
	0.99, \text{ si es descenso geométrico}\\
	0.001, \text{ si es Lundi y Mees}
	\end{cases}$
	\item Número de repeticiones = 250
	\item Tiempo de ejecución (condición de parada) = 60000 (1 minuto por instancia)
\end{itemize}

Los resultados de la ejecución se pueden ver en la Tabla~\ref{tbl:resultados}. En amarillo se muestra el mejor valor de la función objetivo de cada instancia.\\

\begin{table}[htbp!]
	\centering
	\caption{Resultados de la ejecución del recocido simulado}
	\label{tbl:resultados}
	\begin{tabular}{@{}cccccc@{}}
		\toprule
		& \multicolumn{4}{c}{Función de enfriamiento}                                                                                                                                                                                                               &                         \\ \cmidrule(lr){2-5}
		\multirow{-2}{*}{Instancia} & \begin{tabular}[c]{@{}c@{}}Descenso \\ geométrico\end{tabular} & \begin{tabular}[c]{@{}c@{}}Criterio de\\ Boltzmann\end{tabular} & \begin{tabular}[c]{@{}c@{}}Esquema \\ de Cauchy\end{tabular} & \begin{tabular}[c]{@{}c@{}}Lundi \\ y Mees\end{tabular} & \multirow{-2}{*}{Media} \\ \midrule
		1                           & \cellcolor[HTML]{FFFE65}818.9541                               & 883.8607                                                        & 939.1282                                                     & 891.3631                                                & 883.3266                \\
		2                           & 1186.6325                                                      & 1438.0429                                                       & 1000.0160                                                    & \cellcolor[HTML]{FFFE65}974.8233                        & 1149.8787               \\
		3                           & 1152.2130                                                      & \cellcolor[HTML]{FFFE65}996.1984                                & 1103.6910                                                    & 1198.2770                                               & 1112.5949               \\
		4                           & 1026.4867                                                      & 1009.7156                                                       & 1042.9046                                                    & \cellcolor[HTML]{FFFE65}972.5211                        & 1012.9071               \\
		5                           & 1018.6530                                                      & 1033.4533                                                       & \cellcolor[HTML]{FFFE65}889.6963                             & 922.5584                                                & 966.0903                \\
		6                           & 1191.6062                                                      & 1049.2819                                                       & \cellcolor[HTML]{FFFE65}905.0196                             & 1179.4797                                               & 1081.3469               \\
		7                           & 1142.4889                                                      & 1140.0351                                                       & 1472.7656                                                    & \cellcolor[HTML]{FFFE65}914.2835                        & 1167.3933               \\
		8                           & 1077.5733                                                      & 1278.6855                                                       & 1149.6770                                                    & \cellcolor[HTML]{FFFE65}890.9676                        & 1099.2259               \\
		9                           & 1126.6379                                                      & \cellcolor[HTML]{FFFE65}920.1679                                & 971.2918                                                     & 922.4024                                                & 985.1250                \\
		10                          & 1158.27774                                                     & 1130.5921                                                       & 1121.0358                                                    & \cellcolor[HTML]{FFFE65}935.7108                        & 1086.4042               \\ \midrule
		Media                       & 1089.9524                                                      & 1088.0034                                                       & 1059.5226                                                    & 980.2387                                                & \multicolumn{1}{l}{}    \\ \bottomrule
	\end{tabular}
\end{table}

Se puede observar que en la primera instancia se obtiene mejor resultado con el método del descenso geométrico, en las instancias 3 y 9, el mejor método es el criterio de Boltzmann, en las instancias 5 y 6 se obtiene la mejor solución con el esquema de Cauchy, en las instancias 2, 4, 7,8 y 10 se obtiene la mejor solución con el método de Landi y Mees. Es decir, con el método del descenso geométrico se obtiene sólo 1 mejor instancia, con el criterio de Boltzmann y el esquema de Cauchy se obtienen 2 instancias y con el método de Landi y Mees se obtienen 5 instancias. Además, este método tiene la mejor media de tiempos. Parece que este método es el mejor de los cuatro (con los parámetros definidos anteriormente).\\

Necesitamos conocer cuál es mejor valor de $\beta$ que minimiza el valor de las instancias. Para ello, repetimos el proceso anterior para distintos parámetros de $\beta$, desde $1$ hasta $10^{-10}$. Los resultados se pueden ver en la Tabla~\ref{tbl:resultados_landi}.\\

Se puede observar que no existe un valor claro para el parámetro $\beta$, ya que los valores más bajos se encuentran repartidos entre varios valores del parámetro, por lo que el método del recocido simulado no parece muy adecuado para este problema.\\

\begin{landscape}
	\vspace*{\fill}
	\topskip0pt    
	\begin{table}[htbp!]
		\centering
		\caption{Resultados de la ejecución del recocido con el método simulado con la función de enfriamiento de Lundi y Meer para distintos valores del parámetro $\beta$}
		\label{tbl:resultados_landi}
		\resizebox{\linewidth}{!}{%
			\begin{tabular}{@{}cccccccccccc@{}}
				\toprule
				Instancia & $\beta = 1$                      & $\beta = 0.1$                    & $\beta = 0.01$                   & $\beta = 0.001$                   & $\beta = 10^{-4}$ & $\beta = 10^{-5}$ & $\beta = 10^{-6}$ & $\beta = 10^{-7}$                & $\beta = 10^{-8}$                & $\beta = 10^{-9}$                & $\beta = 10^{-10}$ \\ \midrule
				1         & \cellcolor[HTML]{FFFE65}949.7473 & 1021.4972                        & 1024.6271                        & 1009.8923                         & 992.7465          & 1020.6417         & 953.4114          & 1025.7179                        & 1008.4858                        & 1047.6161                        & 835.7937           \\
				2         & 977.4038                         & 1041.2196                        & 953.5492                         & 871.7219                          & 957.7341          & 1053.8826         & 915.0831          & 1004.9098                        & \cellcolor[HTML]{FFFE65}857.8240 & 972.4778                         & 1126.9264          \\
				3         & \cellcolor[HTML]{FFFE65}932.5017 & 991.9547                         & 982.9008                         & 1210.2534                         & 1029.4697         & 1069.4225         & 1065.4087         & 986.1165                         & 1009.0223                        & 947.7720                         & 1025.0647          \\
				4         & 1078.2304                        & 1126.0297                        & 1399.8304                        & \cellcolor[HTML]{FFFE65}1045.2218 & 1170.5624         & 1041.6199         & 1123.5200         & 1102.0909                        & 1109.4818                        & 941.5377                         & 1101.3982          \\
				5         & 833.1772                         & 973.9489                         & 947.5265                         & 893.9518                          & 895.8671          & 893.9518          & 831.0142          & 977.6251                         & 970.4960                         & \cellcolor[HTML]{FFFE65}815.7073 & 882.6927           \\
				6         & 1065.6318                        & 1112.4270                        & \cellcolor[HTML]{FFFE65}998.8739 & 1069.9545                         & 1166.3598         & 1409.8105         & 1217.6866         & 1037.1978                        & 1048.5877                        & 1151.3748                        & 1120.7170          \\
				7         & 1123.1254                        & 1321.5950                        & \cellcolor[HTML]{FFFE65}968.7751 & 1075.8063                         & 1172.2820         & 952.4956          & 1021.7032         & 1138.3258                        & 1225.1920                        & 1060.1519                        & 1082.4603          \\
				8         & 1109.0940                        & 1198.6742                        & 995.1892                         & 1077.5957                         & 1000.3546         & 1125.2670         & 939.0940          & \cellcolor[HTML]{FFFE65}847.4034 & 920.3130                         & 1020.7196                        & 990.3907           \\
				9         & 1022.8755                        & \cellcolor[HTML]{FFFE65}828.5532 & 945.0845                         & 929.0698                          & 903.3034          & 1205.3894         & 982.8460          & 1092.2014                        & 1640.6625                        & 1141.2997                        & 985.7689           \\
				10        & 997.0576                         & 910.6026                         & 922.0289                         & 931.8197                          & 1013.4661         & 1016.7859         & 886.9578          & \cellcolor[HTML]{FFFE65}834.5603 & 1135.09097                       & 905.7793                         & 1082.0464          \\ \bottomrule
			\end{tabular}%
		}
	\end{table}
	\vspace*{\fill}
\end{landscape}

\section{Conclusiones}

En este caso práctico hemos visto cómo implementar una de las metaheurísticas vistas en clase: el recocido simulado. Hemos visto que existen diversos métodos para establecer el enfriamiento, dando todos ellos a distintos resultados. Esto hace que sea indispensable probar con distintas metaheurísticas y escoger la que mejor resultados obtenga para un problema concreto.


\end{document}